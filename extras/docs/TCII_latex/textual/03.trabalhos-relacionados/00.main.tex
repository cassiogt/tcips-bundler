\chapter{Estado da arte}
\label{cap:trabalhos-relacionados}

Em estudos recentes existem algumas propostas para mecanismos de detecção e/ou prevenção de intrusão em \gls{sdn}. Isto se deve basicamente por possuir um controle centralizado e uma visão global da rede, o que as torna eficientes na detecção e reação a intrusos maliciosos. A programabilidade do protocolo OpenFlow permite um gerenciamento mais dinâmico dos fluxos nos comutadores da rede. Esta característica permite o seu uso na área de segurança de redes e vem sendo abordada em diferentes trabalhos que são abordados nesta seção.

\section{Soluções de IDS}

O OpenSAFE, ou \textit{Open Security Auditing and Flow Examination}, abordado por Ballard, Rae e Akella (2010)\nocite{Ballard:2010}, é uma proposta de solução para direcionamento de tráfego à altas taxas de transmissão para propósitos de monitoramento. OpenSAFE pode tratar diversas entradas de rede e gerir o tráfego de tal forma que este pode ser usado por diversos serviços enquanto filtra pacotes na linha. 
OpenSAFE possui três componentes importantes um conjunto de abstrações de \textit{design} para discutir sobre o fluxo de tráfego na rede; uma linguagem de políticas para facilmente especificar e gerenciar rotas chamada ALARMS (A Language for Arbitrary Route Management for Security); e um componente OpenFlow que implementa a política. 

OpenNetMon \cite{Adrichem:2014} é outra abordagem para aplicação de monitoramento de rede na plataforma OpenFlow. Este trabalho implementa um monitor de fluxo para entregar uma entrada refinada para a engenharia de tráfego. Beneficiado das interfaces OpenFlow, que permitem a consulta de estatísticas a partir do controlador, os autores propuseram uma maneira precisa de medir o \textit{throughput} por fluxo, atraso e perdas de pacotes.

Trabalhos como OpenSAFE \cite{Ballard:2010} e OpenNetMon \cite{Adrichem:2014} propõem um serviço de monitoramento da rede para eficientemente coletar estatísticas e detectar atividades maliciosas. No entanto, essas obras não vão além do estágio de detecção e não são capazes de fornecer uma análise mais aprofundada e contramedidas para ataques. A natureza de "detectar" e "alertar" das soluções, exige interação humana para inspecionar os alertas e tomar ações manualmente, não podendo assim, responder à ataques de forma rápida.


No trabalho de Shin e Gu (2012)\nocite{Shin:2012} foi proposto o CloudWatcher para resolver o problema de detecção em redes \textit{cloud} grandes e dinâmicas. Um \textit{framework} para manipular fluxos de rede para nós de segurança onde dispositivos de redes pré instalados possam inspecionar os pacotes, garantindo assim, que todos os pacotes sejam inspecionados. Basicamente o CloudWatcher pode ser realizado como um aplicativo ligado ao controlador de rede, e possui três componentes principais: dispositivo gerenciador de políticas e gestão das informações de dispositivos de segurança, um gerador de regras de roteamento para criar regras para cada fluxo, e um aplicador da regra de fluxo ao comutador. Esta abordagem também permite a implantação de serviços através de \textit{scripts}. No entanto, também não discute contramedidas para atividades maliciosas, mas apenas fornece os serviços de monitoramento.


Zhang (2013)\nocite{Zhang:2013}, aborda um método de contagem de fluxo adaptivo para detecção de anomalias, que  provê um eficiente mecanismo para detecção de anomalias a um baixo custo. Em sua metodologia, uma abordagem dinâmica é obtida através da atualização de regras para reunir as estatísticas e detectar anomalias de acordo com a contagem de tráfego na rede. Os fluxo são agregados e predição linear é utilizada para prever o valor da próxima contagem de fluxo. Desta forma, elimina-se a necessidade de monitorar cada pacote recebido, diminuindo a sobrecarga de monitoramento do controlador, porém, este trabalho também não provê ações de contramedida para proteger dos ataques detectados.


Braga, Mota e Passito (2010)\nocite{Braga:2010} apresentam uma implementação leve, baseada em fluxo para detecção de ataques \gls{ddos}. Este método consiste em monitorar \textit{switches} de uma rede durante intervalos predeterminados de tempo. Durante esses intervalos, são extraídas características de interesse das tabelas de fluxo de todos os \textit{switches}. Cada amostra é então enviada para um módulo classificador que vai indicar, através de algoritmo utilizando técnica de mapas auto-organizáveis (\gls{som}) \cite{Kohonen:1990}, se a informação corresponde ao tráfego normal ou à um ataque. Este trabalho é mais leve comparado aos outros que podem exigir processamento pesado, a fim de extrair a informação característica necessária para a análise de tráfego. No entanto, este documento fornece ênfase apenas a ataques \gls{ddos} e além disso, não fornece contramedida correspondente ao ataque.

Jankowsky e Amanowicz (2015)\nocite{JankowskyAmanowicz:2015}  também abordam um conceito de classificação de fluxo, com base em informações do cabeçalho da camada de transporte, utilizando redes neurais artificiais. Neste modelo, um \textit{testbed} é utilizado para gerar classes de fluxo benignas e maliciosas. Esse fluxo é amostrado e armazenado na memória do controlador \textit{OpenDaylight}. Paralelamente, um cliente coleta estatísticas de fluxo, as armazena e coleta as informações necessárias para posterior classificação, que é realizada utilizando mapas auto-organizáveis Kohonem. Essa abordagem é interessante por prover a detecção de diferentes classes de ataques a um baixo \textit{overhead} da rede, porém possui desvantagens no que diz respeito ao desempenho devido a comparação de todos os pacotes recebidos e, assim como a proposta de Braga, Mota e Passito (2010)\nocite{Braga:2010}, as redes neurais requerem um treinamento prévio com conjuntos de dados artificiais, o que é uma limitação importante na área de \gls{ids}. Além disso, essa proposta também não oferece contramedidas correspondentes aos ataques.

\section{Soluções de IPS}

O IPSFlow \cite{Nagahama:2012} propõe um mecanismo de bloqueio automático de tráfego malicioso utilizando o protocolo OpenFlow. A aplicação atua sobre o controlador para gerenciar e armazenar as regras que definem o encaminhamento dos fluxos na rede baseadas nas definições de segurança. Ao receber um pacote encaminhado pelo \textit{switch}, o controlador consulta o aplicativo IPSFlow para verificar a existência de regras para a captura e análise do tráfego recebido. Caso esteja marcado para ser analisado, pode ser enviado para o destinatário e uma cópia enviada para análise em um \gls{ids} externo. Caso o \gls{ids} conclua que se trata de um fluxo malicioso, o tráfego passa a ser bloqueado no \textit{switch}. Nesta abordagem o tráfego é duplicado para análise no \gls{ids}, gerando fluxos novos na rede. Além disso, os fluxos são analisados de maneira seletiva, havendo grande possibilidade de não inspecionar fluxos maliciosos durante a seleção, já que em um ataque \gls{dos} todos os campos são similares aos benignos.

Avant-Guard \cite{Shin:2013} é apresentado como uma extensão \gls{sdn}, uma implementação em dois módulos: de migração de conexão e de disparo de atuação. Este trabalho é eficiente para filtrar conexões \gls{tcp} incompletas, onde apenas requisições de fluxo que completam o \textit{handshake} vão para o plano de controle. Conexões \gls{tcp} são mantidas pelo módulo de migração de conexão para evitar ameaças de saturação \gls{tcp} (\textit{SYN Flooding}). O módulo de disparo de atuação permite ao plano de dados informar o \textit{status} da rede e ativar uma regra de fluxo específica baseadas em condições pré-definidas. Essa pesquisa melhorou a robustez do sistema \gls{sdn} e fornece recursos adicionais ao plano de dados baixando assim o \textit{overhead} da rede. Este trabalho no entanto é eficiente para casos de ataques de saturação, não absorvendo ataques com \textit{handshake} completo.


Xing \textit{et al.} (2013) \nocite{Xing:2013} propôs um trabalho chamado SnortFlow, que consiste em um \gls{ips} em ambiente de nuvem baseado no analisador de tráfego Snort \cite{Roesch:1999}. O analisador Snort é instalado no domínio de gerência do \textit{hypervisor} XEN, que por sua vez é conectado ao \textit{switch} ligado às máquinas virtuais para inspecionar o tráfego entre elas. Esse trabalho focou basicamente na análise de desempenho, não detalhando características do tráfego e a análise que estava sendo realizada. Além disso, o Snort só consegue analisar o tráfego entre as máquinas, para uma visão global da rede seria necessário alguma forma de sincronização com o controlador.


O NICE proposto por Chung \textit{et al.} (2013)\nocite{Chung:2013}, é uma solução IDS/IPS para SDN que implementa uma análise do tráfego para a construção de um gráfico de ataques e posteriormente gerar dinamicamente contramedidas adequadas em ambientes na nuvem. Este trabalho utiliza a teoria dos grafos para gerar um gráfico de vulnerabilidade e escolher uma solução otimizada na decisão da contramedida. Este modelo porém, é lento na geração do gráfico de ataque para a topologia não sendo prático em uma rede dinâmica.


Lopez \textit{et al.} (2014)\nocite{Lopez:2014} propõem o sistema BroFlow, um sistema \gls{ips} que utiliza a ferramenta de análise de tráfego Bro \cite{Sommer:2010} para inspecionar os pacotes. Esta ferramenta possui sensores distribuídos em pontos estratégicos da rede e emitem alertas quando uma anomalia é detectada. A informação é enviada à um controlador OpenFlow que aciona uma contramedida para bloquear o ataque de maneira global. Este sistema no entanto é baseado em assinatura, não sendo eficiente em redes de altas taxas de transmissão, além disso, possui um problema de otimização da localização dos sensores na rede.

%
Wang, He e Su (2015)\nocite{Wang:2015} proveem em seu trabalho o suporte de funcionalidades mais complexas ao comutador OpenFlow através de \textit{middleboxes} \cite{RFC3234}. Cada \textit{switch} detecta e previne atividades maliciosas através do \textit{middlebox} local e envia alertas para controlador. O controlador por sua vez possui somente a responsabilidade de prover a atualização dos \textit{middleboxes}. Esta abordagem é interessante pois reduz a computação e a comunicação no controlador centralizado porém há a necessidade de equipamentos de rede mais robustos para os dispositivos de rede.

%O Snortik \cite{Fagundes16} propõe uma integração do \gls{ids} SNORT \cite{Roesch:1999} e o sistema de \textit{firewall} do MikroTik RouteOS \cite{mikrotik16}. Este trabalho é dividido em três etapas: Detecção - responsável por identificar e reportar ataques, realizado pela ferramente SNORT; Extração e Conversão - foco do trabalho Snortik, tem por finalidade capturar os registros do \gls{ids} SNORT, extrair as informações e utilizá-las na formatação de regras de proteção; e Aplicação - onde é realizada a aplicação das regras formatadas pelo sistema intermediário no sistema de \textit{firewall} do MikroTic RouteOS. Este sistema provê a proteção contra violações às políticas de uso aceitável da rede (\gls{aup}) e ataques PING \textit{flood}, porém sua solução não aborda conceitos de \gls{sdn}, não possibilitando uma proteção global na rede.

\begin{table}[H]
  \centering
  %\captionof{figure}[tab:trabalhos]{Comparativo entre os trabalhos estudados}
  \caption{Comparativo entre os trabalhos estudados}
  \begin{tabular}{|c|c|c|c|c|} \hline
	\textbf{Trabalho} & \textbf{IDS} & \textbf{IPS} & \textbf{Tipo Ataque} & \textbf{Fonte de Coleta} \\ \hline
	Ballard, Rae, Akella (2010)  & Sim & Não & N/A & Tráfego\\ \hline
	Adrichem, Doerr, Kuipers (2014) & Sim & Não & N/A & Contadores\\ \hline
	Shin, Gu (2012)     & Sim & Não & N/A & Tráfego\\ \hline
	Zhang (2013)    & Sim & Não & N/A & Contadores\\ \hline
	Braga, Mota, Passito (2010)    & Sim & Não & DDoS & Contadores\\ \hline
	Jankowsky, Amanowicz (2015) & Sim & Não & \textit{scan},DoS & Tráfego \\ \hline
	Nagahama \textit{et al.} (2012) & Sim & Sim & \textit{scan} & Tráfego\\ \hline
	Shin \textit{et al.} (2013)    & Não & Sim & DoS & Tráfego\\ \hline
	Xing \textit{et al.} (2013)     & Não & Sim & N/A & Tráfego \\ \hline
	Chung \textit{et al.} (2013)     & Sim & Sim & DDoS & Tráfego \\ \hline
	Lopez \textit{et al.} (2014)    & Sim & Sim & DoS & Tráfego \\ \hline
	Wang \textit{et al.} (2015)     & Sim & Sim & N/A & Tráfego \\ \hline
	%Fagundes \textit{et al.} (2016) & Sim & Sim & AUP, Ping \textit{flood} & Tráfego \\ \hline
	Este trabalho & Sim & Sim & \textit{scan} & Contadores \\ \hline
  \end{tabular}
  \label{tab:trabalhos}
  \fonte{\footnotesize{Elaborado pelo autor.}}
\end{table}

\section{Objetivos}
\label{sec:objetivos}

Considerando os trabalhos estudados na literatura, os quais são sumarizados no Quadro \ref{tab:trabalhos}, percebe-se que o uso de OpenFlow para a implementação de segurança mostrou-se bastante promissor. Porém, observou-se que o foco dado por parte destes não implementa recursos de contramedida para ataques. Dos trabalhos analisados, apenas os trabalhos de Nagahama \textit{et al.} (2012)\nocite{Nagahama:2012}, Chung \textit{et al.} (2013)\nocite{Chung:2013}, Lopes \textit{et al.} (2014)\nocite{Lopez:2014} e Wang \textit{et al.} (2015)\nocite{Wang:2015} apresentam as funcionalidades de detecção e prevenção contra intrusos porém utilizam a análise de tráfego em suas implementações.
Os trabalhos de Adrichem, Doerr e Kuipers (2014)\nocite{Adrichem:2014}, Zhang (2013)\nocite{Zhang:2013} e de Braga, Mota e Passito (2010)\nocite{Braga:2010} utilizam, por sua vez, a análise com base em estatísticas utilizando contadores das tabelas de encaminhamento, porém não contemplam \textit{port scan} e não possuem medidas protetivas.

Neste sentido, motivado pelas limitações dos trabalhos discutidos anteriormente, este trabalho de conclusão de curso propõe uma alternativa para completar as metodologias de detecção e prevenção de intrusão já existentes, através do desenvolvimento de um \gls{ips} baseado em anomalia, fazendo uso do protocolo OpenFlow e utilizando como fonte de informação, contadores das tabelas de fluxo presentes nos \textit{switches} \gls{sdn}. A metodologia de desenvolvimento deste trabalho, bem como sua arquitetura serão apresentados no Capítulo \ref{cap:metodologia} deste trabalho.